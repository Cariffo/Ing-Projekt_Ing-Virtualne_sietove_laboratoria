%\documentclass[17pt,compress,aspectratio=1610]{beamer}
%\documentclass[17pt,compress,aspectratio=169]{beamer}
%\documentclass[17pt,compress,aspectratio=149]{beamer}
%\documentclass[17pt,compress,aspectratio=54]{beamer}
%\documentclass[17pt,compress,aspectratio=43]{beamer}
\documentclass[compress,aspectratio=32]{beamer}
\usepackage[slovak]{babel}
\usepackage[T1]{fontenc}
\usepackage[utf8]{inputenc}
\usepackage{lmodern}
\usepackage{color}
\usepackage{colortbl}
\usepackage{amsmath}
\usepackage{amssymb,amsfonts,amscd}
\usepackage{curves,bezier,epic,rotating}
\usepackage{array,hhline}

\pagestyle{empty}


\usetheme{Warsaw}
%\usetheme{Madrid}
%\usetheme{AnnArbor}


%\useoutertheme{infolines}
%\useoutertheme[hooks]{tree}

%\useinnertheme{circles}

%\usecolortheme{crane}
%\usecolortheme{seagull}
%\usecolortheme{wolverine}

\setbeamercovered{transparent}

\newcommand\modra{\color{blue}}
\def\Bcervena{\color{red!90!black}}
\def\Bmodra{\color{blue!90!black}}
\def\Bcierna{\color{blue!10!black}}
\def\Mmodra{\color{blue!30!white}} 
\newcommand\fsectionN[1][|]{\bf{\Large\bf\color{white!80!blue}\sectionN\ #1\ }\color{white!90!black}\large\bf}

\begin{document}



\title{\bf\color{yellow!90!black} 
Elektronické spracovanie a prezentácia dokumentov \\[-.5em]
\color{white!50!blue}\rule{.9\linewidth}{.5pt}}
\subtitle{\color{white!80!blue} skúsenosti }
\author[rb]{\Large\bf Autor Autor\\
\large Fakulta riadenia a informatiky, Žilinská univerzita}
\date{24. 12. 2015 Žilina}


\setbeamercolor{normal text}{fg=black!90!white,bg=white!90!blue}
%-----------------------------------------------------------------------------------------------------------
\begin{frame}[t]
\titlepage
\end{frame}

%----------------------------------------------------------------------------------------
%..... nazov do horneho riadku
\def\sectionN{Elektronické spracovanie a prezentácia dokumentov}
%----------------------------------------------------------------------------------------
\begin{frame}[t]
\frametitle{\fsectionN[]}
{\bf\Bmodra Názor:} 

Úplne postačí, ak vie študent ovládať niektorý z kancelárskych programov 
a semestrálnu, seminárnu alebo aj bakalársku či diplomovú prácu napíše.
\pause

\smallskip

{\bf\Bmodra Musíme uznať, že to je hlboká pravda, prácu napíše.
\pause 

Ale:} 

\begin{itemize}
\item Ako vyzerá táto práca?   
                               
\item Je čitateľná?            
                               
\item Pôsobí estetickým dojmom?

\item Neznepokojuje nás?                  
                                          
\item Nepôsobí na nás rušivo?             
                                          
\end{itemize}
\end{frame}

%..... nazov do horneho riadku
\def\sectionN{ESPD}
%----------------------------------------------------------------------------------------
\begin{frame}[t]
\frametitle{\fsectionN Základy typografie}

{\bf\Bmodra Typografia:}

\begin{itemize}
\item Často sa zanedbáva (hlavne kompozícia strany a dokumentu).
                                                 
\item Za posledných dvadsať rokov výrazný pokrok. 
      
\item Skoro všetka práca na vývoji dokumentov a kníh sa vykonáva na počítačoch.
      
\end{itemize}
\pause

\medskip


\begin{itemize}
\item Nezanedbateľná je aj estetická časť dokumentu.\\
      {\bf Je iba na tvorcovi, či ma cit pre krásu alebo cit pre gýč.}
      
\item Správna kompozícia stránky má svoje pravidlá:
      
  \begin{itemize}
  \item v texte nesmú byť príliš veľké ani príliš malé medzery,
      
  \item správne delenie slov (často sa ignoruje),

  \item sadzobný obrazec 


  \end{itemize}

\end{itemize}
 


\end{frame}

%





%----------------------------------------------------------------------------------------
\begin{frame}[t]
\frametitle{\fsectionN Optický stred}


\begin{block}{Základné pravidlo získané na základe historických skúseností}
Stred plochy sadzby musí byť umiestnený vždy nad~geometrickým stredom strany. 
\end{block}\pause

\begin{center}
\setlength{\unitlength}{.33em}%
\begin{picture}(40,30)
\put(20,0){\put(-20,0){\line(1,0){40}}\put(-20,0){\line(0,1){30}}%
  \put(20,30){\line(0,-1){30}}\put(20,30){\line(-1,0){40}}%
\only<2-3>{\Mmodra\thinlines\curve(0,30,20,0)\curve(0,30,-20,0)}%
\only<2-3>{\Mmodra\thinlines\curve(-20,30,20,0)\curve(20,30,-20,0)%
  \put(0,15){\circle*{1}}\put(0,13){\makebox(0,0)[t]{G}}}%
\only<3>{\modra\thinlines\curve(-10,20,10,20)%
  \put(0,20){\circle*{1}}\put(0,22){\makebox(0,0)[b]{O}}}%
}
\end{picture}
%
\qquad
%
\begin{picture}(20,30)
\put(0,0){\line(1,0){20}}\put(0,0){\line(0,1){30}}%
  \put(20,30){\line(-1,0){20}}\put(20,30){\line(0,-1){30}}%
\only<2-3>{\Mmodra\thinlines\curve(10,30,20,0)\curve(10,30,0,0)}%
\only<2-3>{\Mmodra\thinlines\curve(0,30,20,0)\curve(20,30,0,0)%
  \put(10,15){\circle*{1}}\put(10,13){\makebox(0,0)[t]{G}}}%
\only<3>{\modra\thinlines\curve(5,20,15,20)%
  \put(10,20){\circle*{1}}\put(10,22){\makebox(0,0)[b]{O}}}%
\end{picture}
\end{center}








\end{frame}


%----------------------------------------------------------------------------------------
\begin{frame}[t]
\frametitle{\fsectionN \LaTeX}

\begin{itemize}
\item Primárne je určený na sadzbu odborných a vedeckých textov.           
                                                                           
\item Bez problémov dokáže realizovať aj hladkú alebo zmiešanú sadzbu.     
                                                                           
\item Nie je WYSIWYG (What You See Is What You get) ako programy typu word. 
                                                                           
\item Je nazávislý na OS (Linux, MS Windows, Mac).                         
                                                                           
\item Nie je klikací.                                                       

\end{itemize}

{\bf\Bmodra
Nedá sa používať tak ako word --- sadnem, klikám, 
klikám,~\dots, klikám a uklikám sa možno k~výsledku.}

\end{frame}


%----------------------------------------------------------------------------------------
\begin{frame}[t]
\frametitle{\fsectionN Niektoré chyby pri tvorbe dokumentov}

\begin{itemize}
\item Wordovské rieky.
                                                                           
\item Nedodržanie typografickej úpravy.
                                                                           
\item Neprehľadnosť atď.
                                                                           
\end{itemize}

\smallskip

Príklad nedodržania elementárnych typografických pravidiel.
Tu sa vkladajú obrázkové súbory:

%\includegraphics[width=.49\linewidth]{obr2a.jpg}
\
%\includegraphics[width=.49\linewidth]{obr2b.jpg}

\end{frame}





\def\sectionN{}
%----------------------------------------------------------------------------------------
\begin{frame}[t]
\frametitle{\fsectionN[]}

\Huge

\vfill

\centerline{Ďakujem za pozornosť.}

\vfill


\end{frame}






%
\end{document}

%