\begin{abstract}

\noindent
{\sc Bc. Šišila Andrej:} {\em \nazovpraceEN} [Diploma thesis] 

\noindent
University of Žilina, Faculty of Management Science and Informatics, Department of information networks.
 
\noindent
Tutor:  doc. Ing. Pavel Segeč, PhD.
 
\noindent
Qualification level: Engineer in field Applied network engineering, Žilina: 

\noindent
FRI ŽU in Žilina, 2018 p. \pageref{LastPage}

\bigskip

The main idea of this thesis is the deployment of a virtual network laboratory into learning process in the Department of information networks.

In the first part are analyzed the tools for network virtualization. These are compared according to chosen criteria, by which a specific solution is selected.

In the second part is described the installation and adjustments of the selected tool which extendeded its functions and corrected some of its flaws. In the end is described the way of server administration.

The third part says about the analysis of the learning topics for selected subjects in the Department of information networks.

The fourth part explains the acquiring and testing of virtual devices. Based on the testing, appropriate devices for specific subjects are selected.

In the last part is described the deployment of the virtual network laboratory into the learning process for specific topics taught in chosen subjects in the Department of information networks.

Keywords: virtualization, laboratory, EVE-ng, GNS3, KVM, Linux

\end{abstract}