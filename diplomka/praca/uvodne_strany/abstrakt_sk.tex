\begin{abstract}

\noindent
{\sc Bc. Šišila Andrej:} {\em \nazovpraceSK} [Diplomová práca] 

\noindent
Žilinská Univerzita v Žiline, Fakulta riadenia a informatiky, Katedra informačných sietí.

\noindent  
Vedúci: doc. Ing. Pavel Segeč, PhD.
 
\noindent  
Stupeň odbornej kvalifikácie: Inžinier v odbore Aplikované sieťové inžinierstvo, Žilina. 

\noindent
FRI ŽU v Žiline, 2018 s. \pageref{LastPage}

\bigskip

Obsahom práce je nasadenie riešenia virtuálneho sieťového laboratória do vyučovacieho procesu na Katedre informačných sietí.

V prvej časti sa zaoberáme nástrojmi pre sieťovú virtualizáciu, ktoré následne porovnáme podľa zvolených kritérií, na základe ktorých vyberieme konkrétny nástroj.

V druhej časti bude opísaná inštalácia vybraného nástroja a úpravy, ktoré rozširovali jeho funkcie a opravovali niektoré z jeho nedostatkov. Nakoniec je opísaný aj spôsob administrácie servera.

Tretia časť je venovaná analýze vyučovaných tém pre vybrané predmety na Katedre informačných sietí.

Štvrtá časť pojednáva o získavaní a testovaní virtuálnych zariadení. Na základe testovania sa vyberú vhodné zariadenia pre vyučované predmety.

V poslednej časti bude popísane nasadzovanie virtuálneho sieťového laboratória do vyučovacieho procesu pre konkrétne témy vyučované na vybraných predmetoch na Katedre informačných sietí.
\\
Kľúčové slová: virtualizácia, laboratórium, EVE-ng, GNS3, KVM, Linux

\end{abstract}