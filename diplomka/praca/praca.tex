\documentclass[a4paper,12pt]{report}
\usepackage[slovak,english]{babel}
\usepackage[T1]{fontenc}             
\usepackage[utf8]{inputenc}    
\usepackage{lmodern}
\usepackage{amsmath}
\usepackage{amssymb,amsfonts,amscd}
\usepackage{array,hhline}
\usepackage{makeidx}
\usepackage{fancyhdr}
\usepackage{graphicx}
\usepackage{listings}
    %%\usepackage{titlepage}
    %%\usepackage{multicol}
\usepackage{eurosym}
\usepackage{url,mathptmx} 
\usepackage[pdftex,unicode,bookmarks=false]{hyperref}

\renewcommand{\baselinestretch}{1.5}  % pre zvascenie riadkovania

\addtolength{\oddsidemargin}{-.5cm}
\addtolength{\evensidemargin}{-2.9cm}   
\addtolength{\topmargin}{0cm}
\addtolength{\textheight}{0pt}
\addtolength{\textwidth}{2.cm}
\addtolength{\textheight}{2.cm}
\newlength{\verbcorr}
\setlength{\verbcorr}{0ex}

\graphicspath{{./obrazky/}}

\newcommand{\nazovpraceSK}{Analýza nástrojov na virtualizáciu sieťových prvkov a ich použitie vo vyučovacom procese}
\newcommand{\nazovpraceEN}{Analysis of the tools for network devices virtualization and their use in learning process}

%%%%%%%%%%%%%%%%%%%%%%%%%%%%%%%%%%%%%%%%%%%%%%%%%

\begin{document}
\selectlanguage{slovak}

\pagestyle{empty}
\pagenumbering{arabic}

% Titulna strana, Abstrakt SK, Abstrakt EN, Prehlasenie
\begin{titlepage}

\phantom.

\bigskip

\begin{center}
{\sc\LARGE Žilinská Univerzita v Žiline}
\medskip

{\sc\Large Fakulta riadenia a informatiky}

\vfill\vfill\vfill\vfill

{\sc\LARGE Diplomová práca}

\medskip

{\large Študijný odbor: {\bf Aplikované sieťové inžinierstvo}}
\end{center}


\vfill\vfill\vfill\vfill


\phantom.\hfill
%\begin{minipage}{10cm}
\begin{center}
{\large\bf Andrej ŠIŠILA}

\medskip

{\large\bf \nazovpraceSK}

\medskip

Vedúci: {\bf doc. Ing. Pavel Segeč, PhD.}

\medskip
\bigskip

Registračné číslo: 322/2017

\medskip

Ministerské číslo práce: 28360320182322

\medskip

Máj 2018

\end{center}
%\end{minipage}
\hspace{1.7cm}\phantom.

\vspace{2.9cm}

\phantom.
\end{titlepage}


%--------------------------------------------------------------------------------------
%%% slovensky abstrakt

\begin{abstract}

\noindent
{\sc Šišila Andrej:} {\em \nazovpraceSK}
[Diplomová práca] 

\noindent
Žilinská Univerzita v~Žiline,  
Fakulta riadenia a informatiky,  
Katedra informačných sietí.

\noindent  
Vedúci: doc. Ing. Pavel Segeč, PhD.
 
\noindent  
Stupeň odbornej kvalifikácie:
Inžinier v odbore Aplikované sieťové inžinierstvo, Žilina. 

\noindent
FRI ŽU v Žiline, 2017 TODO s.

\bigskip

Obsahom práce je ... TODO


\end{abstract}


%--------------------------------------------------------------------------------------
%%% anglicky abstrakt


\selectlanguage{english}
\begin{abstract}

\noindent
{\sc Šišila Andrej:} {\em \nazovpraceEN}
[Diploma thesis] 

\noindent
University of Žilina,  
Faculty of Management Science and Informatics, 
Department of information networks.
 
\noindent
Tutor:  doc. Ing. Pavel Segeč, PhD.
 
\noindent
Qualification level:
Engineer in field Applied network engineering, Žilina: 

\noindent
FRI ŽU v~Žiline, 2017 TODO p.

\bigskip

The main idea of this ... TODO

\end{abstract}
\selectlanguage{slovak}


%%%%%%%%%%%%%%%%%%%%%%%%%%%%%%%%%%%%%%%%%%%%%%%%%%%%%%%%%%%%%%%%%%%%%%%
\newpage

\centerline{\bf Prehlásenie}

\vspace{2em}

\noindent
Prehlasujem, že som túto prácu napísal samostatne a že som uviedol 
všetky použité pramene a literatúru, z~ktorých som čerpal. 

\vspace{2em}

\noindent
V Žiline, dňa XX.YY.ZZZZ TODO
\hfill
Andrej Šišila



%\pagestyle{myheadings} % tento riadok mi pridaval cislo 1 do praveho horneho rohu na strane "Prehlasenie"

% Predpokladam, ze subor "title" vygeneruje 4 strany, preto tymto cislom zacinam cislovanie od strany "Prehlasenie"
\setcounter{page}{4}

{\setlength{\parskip}{1pt plus 1pt}

\markboth{}{}

\tableofcontents

\newpage

%\vspace{0pt plus 2cm}

\listoffigures

%\vspace{0pt plus 2cm}

\listoftables
}

\markboth{}{}

\clearpage

%%%%%%%%%%%%%%%%%% obsah - koniec

%%%%%%%%%%%%%%%%%% kapitoly

\chapter*{Úvod}
\addcontentsline{toc}{chapter}{Úvod}

Virtualizácia sa stáva vo svete čoraz populárnejšou. Využíva sa v rôznych oblastiach napríklad v tzv. Cloud Computing alebo Software Defined Networking. Virtualizáciou sa zaberá aj Katedra informačných sietí Žilinskej univerzity v Žiline. Jedným z projektov, kde sa virtualizácia využíva, je projekt virtuálneho sieťového laboratória. 

Mojou úlohou je preskúmať existujúce riešenia v oblasti virtuálnych sieťových laboratórii a porovnať ich podľa vopred stanovených kritérií. Na základe toho z nich vyberiem jedno riešenie, ktorým sa budem v práci ďalej zaoberať. Pre vybrané sieťové laboratórium následne vypracujem návody na inštaláciu, úpravu, používanie a nasadenie do infraštruktúry katedry. Následne analyzujem kompatibilitu nástroja s rôznymi zariadeniami. Vybrané zariadenia otestujem, do akej miery vyťažujú systémove zdroje, aby som im mohol nastaviť primerané parametre. Potom si vyberiem predmety, na ktoré má byť tento nástroj použítý a analyzujem technológie vyučované na týchto predmetoch. Nakoniec zistím, ktoré zariadenia sú pre daný predmet vhodné.

Virtuálne sieťové laboratórium bude nástrojom, ktorý zefektívni a skvalitní výučbu sieťových technológii vo vybraných predmetoch na Katedre informačných sietí.

Tému diplomovej práce som si vybral predovšetkým preto, aby som pomohol dosiahnuť tento cieľ: skvalitniť a zefektívniť vyučovací proces na katedre. Tak budú mať učitelia aj študenti jednotnú platformu pre vyučovanie, ktorá učiteľom umožní jednoducho vytvárať modelové situácie a študentom uľahčí ich pochopenie.
\chapter{Nástroje pre sieťovú virtualizáciu}

Čo je:
EVE-ng, GNS3, Dynamips, VIRL \cite{pes2} \cite{bartsch2}, \cite{berman, wolfram}

Odkiaľ je, čím sa vyznačuje, ako sa ovláda
\chapter{Kritériá testovania}

kompatibilita zariadení, maximálny počet zariadení každého typu, stabilita backendu a frontendu

v každom nástroji otestovať dve zariadenia a zmerať vyťaženie CPU a RAM
\chapter{Výsledky testovania}

Výsledky testovania nástrojov (2 zariadenia z každého; Cisco tam bude určite, to rozbehnú všetky) -> zabaliť do prehľadnej tabuľky
\chapter{Aplikovanie nástrojov vo vyučovacom procese}

Rozpísať, ako sa ktorý nástroj správal pri vypracovávaní úloh z daného predmetu.

\section{Počítačové siete 1}
\subsection{EVE-ng}
\subsection{GNS3}
\subsection{Dynamips}
\subsection{VIRL}

\clearpage

\section{Počítačové siete 2}
\subsection{EVE-ng}
\subsection{GNS3}
\subsection{Dynamips}
\subsection{VIRL}

\clearpage

\section{Projektovanie sietí 1}
\subsection{EVE-ng}
\subsection{GNS3}
\subsection{Dynamips}
\subsection{VIRL}

\clearpage

\section{Projektovanie sietí 2}
\subsection{EVE-ng}
\subsection{GNS3}
\subsection{Dynamips}
\subsection{VIRL}

\clearpage

\section{OKS}
\subsection{EVE-ng}
\subsection{GNS3}
\subsection{Dynamips}
\subsection{VIRL}
\chapter{Záver}

Tu treba zhodnotiť dosiahnuté výsledy a načrtnúť dalšie možné cesty riešenia.




\chapter*{Prílohy}
\addcontentsline{toc}{chapter}{Prílohy}

CD médium obsahuje:

\begin{itemize}
    \item Text bakalárskej práce,
    \item Návody na používanie pre virtualizačný nástroj
    \item Tabuľkový dokument so sumárnym vyhodnotením všetkých otestovaných zariadení pre virtualizačný nástroj
    \item Tabuľkový dokument s prehľadom všetkých vyučovaných technológii na vybraných predmetoch a ich kompatibilitou s vybranými virtuálnymi zariadeniami
\end{itemize}


%%%%%%%%%%%%%%%%%% literatura


%%%%%%%%%%%%%%%%%%%%%%%%%%%%%%%%%%%%%%%%%%%%%%%%%%%%%%%%%%%%%%%%%%%%%%%%%%%%%
\begin{thebibliography}{99}                                \label{literatura}
%\addcontentsline{toc}{section}{Literatúra}
\addcontentsline{toc}{chapter}{Literatúra}

\bibitem{bartsch2}
Bartsch H. J.,
{\it Matematické vzorce}, 
3.~revidované vydání, Praha, Mladá fronta 2000, ISBN 80-204-0607-7.
         
\bibitem{berman}
Berman G. N.,
{\it Zbierka úloh z matematickej analýzy}, 
Bratislava, ŠNTL 1955.
         
\bibitem{pes2} 
Peško, Š., 
{\it Pohodlná optimalizácia reálnych úloh v tabuľkových procesoroch}, 
Slovak Society for Operations Research, 7th international seminar, 
Application of Quantitative Methods in Research and Practice  (2005), 
pp. 29--35, Remata, ISBN 80-225-2079-9.



\bibitem{wolfram}
World of mathematics, A Wolfram Web Resource, 
\url{http://mathworld.wolfram.com/}, 
WolframAlpha -- computational knowledge engine, 
\url{http://www.wolframalpha.com/}.


\end{thebibliography}

	%  Literatura

\end{document}


