\chapter{Nástroje pre sieťovú virtualizáciu}

Čo je:
EVE-ng, GNS3, Dynamips, VIRL 

Odkiaľ je, čím sa vyznačuje, ako sa ovláda
Prieskum dostupných virtualizačných nástrojov + súčasný stav na katedre \cite{pes2} \cite{bartsch2}, \cite{berman, wolfram}

\section{Motivácia}
- Dôvody, prečo použiť virt. laby
  - ieee explore, sci/scio hub - na stiahnutie clankov z ieee exp
  
\section{Porovnávacie kritériá}

- použité ývojové technológie (prog jazyk: backend, frontend)
- podpora zariadení, s ktorými sa na danom predmete pracuje
- podpora technológií jednotlivých zariadení
- typ používateľského rozhrania
- prideľovanie portových čísel zariadeniam
- vzdialený prístup ku zariadeniam (telnet, vnc, rdp)
- správa topológií
  - vytvorenie/úprava/uloženie/odstránenie topológie
  - koľko topológií môže mať jeden používateľ spustených
- možnosť práce viac ľudí naraz na rovnakom projekte
- možnosť prepojiť topológiu so živou sieťou


\section{Dostupné riešenia}

\subsection{Dynamips}

\subsection{WEB-IOU}

\subsection{Cisco VIRL}

\subsection{VIRO}

Projekt Petra Hadača.

\subsection{UNetLab}

\subsection{EVE-ng}

popis + rozdiely rôznych verzií (špeciálne pre EVE-ng) - PRIMÁRNE! + rozdiely vo funkcionalitách podľa tabuľky na oficiálnej stránke EVE-ng.

\subsection{GNS3}

- podporný nástroj

\section{Vyhodnotenie}

Porovnanie GNS3 a EVE-ng.

Odôvodnenie, prečo som si vybral práve EVE-ng.