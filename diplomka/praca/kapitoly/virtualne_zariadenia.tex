\chapter{Virtuálne zariadenia}

\section{Získavanie}

Zariadenia boli získavané z rôznych zdrojov. Predovšetkým boli použité zariadenia, ktoré sa už na katedre používali.

Spomenúť typy zariadení:
  - Dynamips
  - Cisco IOL/IOU
  - QEMU
  
Zoznam všetkých funkčných zariadení v nástroji EVE-ng je k dispozícii na CD médiu.

\section{Testovanie}

\subsection{Testovanie spustiteľnosti}

Kompatibilita zariadení.
Kritériá

\subsubsection{Metodika}

\subsubsection{Vyhodnotenie}

Sumárny prehľad.ods

\subsection{Testovanie systémových požiadaviek}

Kritériá

\subsubsection{Metodika}

Testované skriptom.


Pre vybrané zariadenia boli merané tieto veličiny:
\begin{itemize}
\item vyťaženie CPU
\item nároky na operačnú pamäť
\item vyťaženie pevného disku
\end{itemize}

\subsubsection{Vyhodnotenie}

benchmarking.zip

Maximálny počet zariadení každého typu je možné odhadnúť sčítaním ich systémových požiadaviek. Rovnako odhadneme aj maximálny počet spustených topológii daného typu.

\subsection{Testovanie technológii}

Kritériá.

Zatiaľ boli testované podporované technológie iba na Cisco zariadeniach.

\subsubsection{Metodika}

Testované skriptom

\subsubsection{Vyhodnotenie}

Ktoré zariadenia podporujú technológie vyučované na vybraných predmetoch.

Vyučované technológie.ods

\section{Kritériá testovania}