\chapter{Virtuálne zariadenia}
\label{chap:virt_zariadenia}

Dôležitú súčasť virtuálneho laboratória tvoria aj jeho zariadenia. Po analýze vyučovania sme mohli začať s ich získavaním a testovaním.




\section{Získavanie}

Zariadenia boli získavané z rôznych zdrojov. Predovšetkým boli použité zariadenia, ktoré sa už na katedre používali.

Ako už bolo naznačené v časti \ref{chap:adresarova_struktura}, EVE-ng podporuje rôzne typy zariadení. Patria medzi ne zariadenia typu Dynamips, Cisco IOL/IOU a QEMU.

Zoznam všetkých funkčných zariadení v nástroji EVE-ng je k dispozícii na CD médiu.




\subsubsection{Metodika}

Pri získavaní zariadení sme vychádzali zo zariadení, ktoré sa na katedre používajú a zo zoznamu zariadení, ktoré nástroj EVE-ng podporoval. Vybrané zariadenia následne prechádzali testovacími krokmi, ktoré sú bližšie popísané v ďalších častiach tejto kapitoly.

Všetky zozbierané zariadenia sú uložené v adresári \\ \texttt{/opt/unetlab/addons/rozne\_zariadenia}.




\section{Testovanie}
\label{chap:testovanie_zariadeni}

Testovanie zariadení bolo vykonané, aby sa zaistila kvalita a plynulosť vyučovacieho procesu, ako aj predvídateľnosť a replikovateľnosť behu jednotlivých zariadení a topológii.




\subsection{Testovanie spustiteľnosti}

Ako prvé bola testovaná spustiteľnosť zariadení. Na základe nej sa zistilo, či je zariadenie schopné zapnúť a konfigurovať ho pomocou vzdialeného prístupu.



\subsubsection{Metodika}

Najpr bolo potrebné pridať zariadenie na server. Môžeme na to použiť návody buď z oficiálnej EVE-ng stránky \cite{eve_ng_howtos}. Návody pre niektoré zariadenia sú k dispozícii aj na CD v adresári \emph{eve-ng}.

Vybrané zariadenia sme následne pridávali do predpripravenej topológie v EVE-ng. Následne sme vybrali jedno zariadenie, ktoré sme následne spustili. Ak sa zariadenie nespustilo, skúšali sme zistiť príčinu a vykonávali rôzne úpravy, akými by sme zariadenie v topológii spustili napr. modifikovať súbor, z ktorého sa zariadenie spúšťalo, opraviť oprávnenia, premenovať súbor do správneho formátu, upraviť systémové parametre na zariadenia a pod.

Ak sa zariadenie spustilo, skúsili sme sa pripojiť na jeho konzolu. Keď sa na ňu nedalo pripojiť, zariadenie sme zastavili a zmenili protokol na vzdialený prístup; obvykle stačilo vyskúšať protokoly \emph{telnet} a \emph{vnc}. V prípade, že zariadenie je v poriadku, mali by sme aspoň jedným z týchto protokolov dostať textový resp. grafický výstup konzoly na zariadení.

Ak sme sa nakoniec pripojili ku konzole zariadenia, sledovali sme jeho spúšťanie. Čakali sme na dokončenie spúšťania. V prípade, že sa zariadenie úspešne spustilo, prihlásili sme da doň predvolenými prihlasovacími údajmi, ak to bolo potrebné, a vyskúšali sme, či konzola reaguje na vstup z klávesnice. Ak konzola reagovala, zariadenie zostalo uložené na serveri. Ak sme ku zariadeniu nevedeli zistiť prihlasovacie údaje, umiestnili sme ho do osobitného adresára.

Ak sa ani po týchto úkonoch zariadenie nespustilo, odstránili sme ho zo servera. Týmto spôsobom sme získali množinu spustiteľných zariadení v EVE-ng topológii.

\subsubsection{Vyhodnotenie}

Výsledky testovania spustiteľnosti zariadení sú zhrnuté v súbore \emph{sumarny\_prehlad\_podporovanych\_zariadeni\_vo\_virtualnych\_sietovych\_nastrojoch.ods}. Súbor obsahuje viacero stĺpcov

{\huge TODO - POPÍSAŤ STĹPCE}

Zariadenie	Platforma	Názov súboru	Spôsob virtualizácie	Predvolené prihlasovacie údaje	Spôsob pripojenia	Úspešné spustenie	Poznámky


V stĺpci \emph{Poznámky} obsahuje aj špecifický popis správania sa daného zariadenia, popr. problémy pri používaní zariadenia a ich možné riešenie.

\subsection{Testovanie systémových požiadaviek}
\label{chap:testovanie_zariadeni_benchmark}

Kritériá

\subsubsection{Metodika}

Testované skriptom - upravený nástroj \emph{dstat}.


Pre vybrané zariadenia boli merané tieto veličiny:
\begin{itemize}
\item vyťaženie CPU
\item nároky na operačnú pamäť
\item vyťaženie pevného disku
\end{itemize}

\subsubsection{Vyhodnotenie}

benchmarking.zip

\subsection{Testovanie technológii}
\label{chap:testovanie_technologii}

V tejto časti využijeme poznatky z kapitoly \ref{chap:analyza_vyucovania} a použijeme ich na testovanie podporovaných technológii vybraných zariadení.

Kritériá.

Zatiaľ boli testované podporované technológie iba na Cisco zariadeniach.

\subsubsection{Metodika}

Testované skriptom

\subsubsection{Vyhodnotenie}

Ktoré zariadenia podporujú technológie vyučované na vybraných predmetoch.

Vyučované technológie.ods