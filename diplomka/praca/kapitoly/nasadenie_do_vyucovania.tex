\chapter{Nasadenie do vyučovania}
\label{chap:nasadenie_do_vyucovania}

Rozpísať, ako sa ktorý nástroj správal pri vypracovávaní úloh z daného predmetu.

Urobiť napríklad scenár: syslog a snmp server z inych hostov a routrov

Maximálny počet spustených topológii daného typu odhadneme sčítaním ich systémových požiadaviek. Mali by z toho vyplynúť odhad minimálnych systémových požiadaviek servera pri topológiách na vybrané predmety.




\section{Vytváranie topológii v EVE-ng}
\label{chap:nasadenie_klient}

Web rozhranie - 2 módy - Natívny a HTML5.
Integračný balíček je nutný pre Natívny režim. HTML5 mód zabezpečuje integráciu pomocou reverzného proxy servera \emph{Apache Guacamole}, ktorý sa pripája na konzoly zariadení.

HTML5 mód web rozhrania EVE-ng bol však reagoval výrazne pomalšie pri práci s topológiou v provonaní s natívnym módom.

- predvolená vs KIS verzia integračného balíčka -> SSH tunely (pre vzdialený prístup k zariadeniam v topológii).




\section{Používanie EVE-ng}

-popisat
      -vytvorenie topologie
        -z GUI (pridavanie zariadenia)
        -editovaním UNL súboru s topológiou
            -popisat hlavne parametre, ktore sa lisia pri duplikovani textovej reprezentacie danej topologie
        -vyslovit odporucanie na vytvaranie topologii
            -> GUI -> Nodes (nazvy zariadeni) -> UNL (skraslenie topologie upravenim suradnic pre zariadenia) - VID A4 PAPIER "DUPLIKÁCIA TOPOÓGIE V EVE-NG -> ZÁVER"
      -pridelovanie portovych cisel zariadeniam
        -rozsah
        -pridelovanie portovych cisel je sekvencne
      -spustanie zariadeni
        -po jednom
        -vybrana skupina
          -Ctrl+klik -> pravy klik -> Start Selected
          -oznacenie mysou -> pravy klik -> Start selected
        -vsetky zariadenia v topologii
          -More actions -> Start all nodes
      -odhadnut systemove poziadavky pre topologie z vybranych predmetov a na zaklade toho odhadnut minimalne systemove poziadavky servera
      -> začať topológiami z predmetov, na ktorých bol nástroj nasadený.




\section{Počítačové siete 1 a 2}

- náhrada za nástroj Packet Tracer a doplnok pre nástroj Dynamips/Dynagen

Na predmet Počitačové siete 2 bol použitý fyzický EVE-ng server.

-NASADENIE NA 7. TÝŽĎEŇ 6.4.2018 - PS2
      -IOL smerovače fungovali, až na nastavenie clock rate na sériových rozhraniach, bez chyby. V jednej skupine sa vyskytol problém s jednosmernou PAP autentifikáciou študentského smerovača voči učiteľskému (R\_Ucitel(s4/1)-5R2(s2/1)). Riešenie spočívalo v odstránení používateľa, vypnutí ppp konfigurácie a vypnutí rozhraní na oboch rozhraniach aj na učiteľskom aj na študentskom smerovači. Následne sa konektivita obnovila.

  Pridať aj troubleshooting a vypisy z routrov (ale iba velmi kratke), aby bolo jasne, v com bol problem a ako sme ho riesili (vid otvorene konzoly routrov na ploche 1)

Vypracované topológie:
- Topológia s point-to-point technológiami.

-vyhodnotit dstat csv - (8 topologii = 32 Cisco IOL smerovačov (4*8 + 1 učiteľský) + 16 QEMU zariadení (Alpine Linux x86))
        -> VYHODNOTIT CEZ LIBRE CALC
        
    
    
{\huge TODO - TOTO DAŤ DO KAPITOLY ÚPRAVA NÁSTROJA EVE-NG}
-zistený limit - maximálny počet zariadení obmedzený na 63; 64. zariadenie sa chvíľu tvári, že sa spúšťa, ale po niekoľkých sekundách sa automaticky vypne.

{\huge TODO - AJ TOTO DAŤ DO KAPITOLY ÚPRAVA NÁSTROJA EVE-NG}
-EVE-ng v momentalnom stave vie zatvorit topologiu, ale nevie po zatvoreni jednej topologie (co teraz funguje), pracovat so zariadeniami v inej topologii. Z toho vyplýva, že v EVE-ng môžeme mať spustenú iba jednu topológiu, ktorá je navyše obmedzená na 63 zariadení, 64. sa síce začne spúšťať, ale hneď na to sa vypne.
    ->DOCASNE RIESENIE spociva v rozsireni portoveho rozsahu pre jednu topológiu. To vyrieši problém s obmedzením počtu zariadení v jednej topológii, ale nerieši to problém s nezávislým spúšťaním topológii ako jeden používateľ.



\section{Projektovanie sietí 1}

náhrada/doplnok Dynamips servera (s takym vykonom, t.j. cca 10/15skupin po 10 routroch)




\section{Projektovanie sietí 2}

VMware inštalácia

V EVE-ng boli úspešne dokončené semestrálne práce s využitím technológii VPLS a Seamless MPLS.

\section{CCNP Routing}

\section{CCNP Switching}