\chapter{Nasadenie do vyučovania}

Rozpísať, ako sa ktorý nástroj správal pri vypracovávaní úloh z daného predmetu.

Urobiť napríklad scenár: syslog a snmp server z inych hostov a routrov

Maximálny počet spustených topológii daného typu odhadneme sčítaním ich systémových požiadaviek. Mali by z toho vyplynúť odhad minimálnych systémových požiadaviek servera pri topológiách na vybrané predmety.

\section{Počítačové siete 1 a 2}

- náhrada/doplnok pre nástroj Packet Tracer

Vypracované topológie:
- sumárne laboratórne cvičenie \emph{Packet Tracer Packet Tracer Skills Integration Challenge 8.6.1}
- Topológia s point-to-point technológiami.

\section{Projektovanie sietí 1}

náhrada/doplnok Dynamips servera (s takym vykonom, t.j. cca 10/15skupin po 10 routroch)

\section{Projektovanie sietí 2}

V EVE-ng boli úspešne dokončené semestrálne práce s využitím technológii VPLS a Seamless MPLS.

\section{CCNP Routing}

\section{CCNP Switching}