\chapter{Analýza vyučovania}
\label{chap:analyza_vyucovania}

PER PREDMET: Aké sieťové predmety sa vyučujú, aké nástroje sa na nich používajú, s akými zariadeniami sa pracuje a aké technológie


\section{Vybrané predmety}

- Počítačové siete 1
  - Zariadenia:
    - smerovače: Cisco, Juniper

- Počítačové siete 2
  - Zariadenia:
    - smerovače: Cisco, Juniper

- Projektovanie sietí 1 - PRIMÁRNE!
  - Zariadenia:
    - smerovače: Cisco, Juniper
    - koncové zariadenia: (Linux / Windows) (pre multicast technológie)

- Projektovanie sietí 2 - PRIMÁRNE!
  - Zariadenia:
    - smerovače: Cisco, Juniper, Nokia

- CCNA Security
  - Zariadenia:
    - prepínače: Cisco
    - smerovače: Cisco
    - firewall: Cisco
                
- CCNP Routing \& Switching - PRIMÁRNE!
  - Zariadenia
    - prepínače: Cisco
    - smerovače: Cisco
    
    
    


Ku všetkým predmetom vytvoriť zoznam technológii.




- Počítačové siete 1 (CCNA 1, CCNA 2)
  - TODO - konkrétne technológie, vyučované na predmete

- Počítačové siete 2 (CCNA 3, CCNA 4)
  - aktualny stav: zakladny routing (7200) a switching l2 aj l3 (nevieme zatial)
  - TODO - konkrétne technológie, vyučované na predmete

- Projektovanie sietí 1 - PRIMÁRNE!
  - TODO - konkrétne technológie, vyučované na predmete

- Projektovanie sietí 2 - PRIMÁRNE!
  - TODO - konkrétne technológie, vyučované na predmete

-CCNA Security
  - TODO - konkrétne technológie, vyučované na predmete

- CCNP Routing \& Switching - PRIMÁRNE!
  - routing
  - TODO - konkrétne technológie, vyučované na predmete

\subsection{Počítačové siete 1}

\subsection{Počítačové siete 2}

\subsection{Projektovanie sietí 1}

\subsection{Projektovanie sietí 2}

\subsection{CCNP Routing}

\subsection{CCNP Switching}