\chapter{Záver}

Cieľom práce bolo nasadenie virtuálneho sieťového laboratória do vyučovacieho procesu katedry. Najprv bolo potrebné preskúmať existujúce riešenia pre virtuálne sieťové laboratóriá a porovnať ich na základe zvolených kritérií. Z porovnania vyplynulo, že existujú viaceré virtuálne laboratória, ktorými sa má zmysel ďalej zaoberať: GNS3 a EVE-ng. Nakoniec však bol zvolený druhý menovaný nástroj. Následne bol tento nástroj nainštalovaný na virtuálnu platformu VMware a fyzický server.

Po nasadení nástroja do infraštruktúry katedry boli analyzované vyučované témy vybraných predmetov na Katedre informačných sietí. Táto analýza pomohla pri získavaní virtuálnych zariadení do nástroja EVE-ng. Získané zariadenia boli v nástroji EVE-ng testované na ich spustiteľnosť, systémové požiadavky a podporu vyučovaných tém. Pre analyzované predmety boli vybrané zariadenia, ktoré boli počas nasadenia do vyučovania použité pri vytváraní topológii. Zariadenia boli pre predmety vyberané tak, aby pokryli čo najväčšiu časť vyučovaných tém. Študenti na cvičeniach pod vedením vyučujúceho používali tento nástroj, aby sa overila jeho použiteľnosť v reálnom vyučovacom procese.

Celý proces bol dôsledne a prehľadne dokumentovaný. Dokumentácia podrobne opisuje celý proces práce od inštalácie nástroja EVE-ng a jeho následnej úpravy až po získavanie a testovanie virtuálnych zariadení a nasadenie nástroja do vyučovania. Dokumentácia je prítomná na priloženom CD, ktorého adresárovú štruktúru je možné vidieť v kapitole \ref{chap:cd}.

Na výsledkoch tejto práce sa dá v budúcnosti pokračovať v mnohých smeroch. Práca môže slúžiť ako východiskový bod pri skúmaní ďalších virtuálnych sieťových laboratórií s použitím metodík popísaných v rôznych častiach tejto práce. To by mohlo viesť k vytvoreniu ideálneho riešenia pre virtuálne sieťové laboratórium, ktoré by v sebe kombinovalo výhody nástrojov ViRo2, GNS3 a EVE-ng, pričom jadro by tvoril práve nástroj GNS3. Ideálne sieťové laboratórium by obsahovalo rezervácie topológii z nástroja ViRo2, grafické používateľské rozhranie, jednoduchú škálovateľnosť naprieč viacerými servermi, dodatočnú podporu Docker popr. LXC/LXD kontajnerov a izoláciu používateľov spolu ich súbormi a adresármi. Vo svojej práci som sa snažil splniť aspoň niektoré zo spomenutého stručného zoznamu požiadaviek. V každom prípade nasadenie konkrétneho nástroja do vyučovacieho procesu katedry posunulo vyučovanie na nej na vyššiu úroveň, čo dokazuje, že virtuálne sieťové laboratórium tvorí dôležitú súčasť pri vyučovaní sieťových technológii.