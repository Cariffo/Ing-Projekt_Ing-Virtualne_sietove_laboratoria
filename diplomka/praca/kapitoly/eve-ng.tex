\chapter{EVE-ng}

\section{Inštalácia}

Kde všade je EVE-ng server nainštalovaný a v akej verzii.

\begin{itemize}
    \item VMware Workstation Player - .49
    \item Fyzický server - .50
\end{itemize}

Parametre VMware virtuálky aj fyzického servera. V akom stave je EVE-ng na oboch platformách?

Uvedené tvrdenia platia pre EVE-ng vo vydaní \emph{Community Edition} vo verzii 2.0.3-86.

  - Postup
    - Inštalácia Ubuntu Server 16.03
    - Konfigurácia Ubuntu Server
    - Inštalácia EVE-ng do Ubuntu Server
    - Konfigurácia EVE-ng
      - Obnovenie súborov a adresárov
        - Skripty
        - Zariadenia
        - Databázy
      - Automatizácia zálohovania nástrojom "cron"
      - Cisco IOL/IOU licencia
      - Zabezpečenie servera
  - úprava šablón

\section{Úpravy po inštalácii}

Tu pomôžu názvy súborov v adresári \emph{eve-ng} v repozitári/CD.

\subsection{Linux server}
    Zabezpečenie
  - Systém
  - SSH
  - Web server
  - Databázový server
  
\subsection{EVE-ng}

Ako som modifikoval zdrojový kód EVE-ng.

\subsection{Klientský počítač}

- SSH tunely (pre vzdialený prístup k zariadeniam v topológii).

\section{Administrácia}

\subsection{Adresárova štruktúra}

\subsection{Zálohovanie}

\subsection{Monitorovanie}

\subsubsection{EVE-ng - System status}

\subsubsection{netdata}

\section{Používanie nástroja EVE-ng}

\subsection{Správa používateľov}

\subsubsection{Používateľské role}

\subsection{Vytvorenie topológie}
