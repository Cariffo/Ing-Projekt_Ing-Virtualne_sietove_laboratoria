\chapter*{Úvod}
\addcontentsline{toc}{chapter}{Úvod}

V poslednom čase sa čoraz viac začína využívať virtualizácia v rámci sieťových technológií. Virtualizácia je spôsob, ako jeden fyzický počítač dokáže spúšťať viacero operačných systémov súčasne.

Virtualizácia sa prejavuje napríklad v Cloud Services alebo Software Defined Networking. Virtualizovať môžeme aj sieťové prvky, čo veľmi podobné, ako virtualizovať operačný systém. Virtualizované zariadenia sa ovládajú rovnako, ako fyzické, takže študenti získajú rovnaké skúsenosti.

Spomeňme niektoré výhody virtualizácie:
\begin{itemize}
\item šetrenie nákladov - Namiesto kúpy nového hardvéru, vieme pomocou nástrojov simulovať viacero druhov sieťových prvkov na jednom fyzickom serveri.
\item menšie priestorové požiadavky - Nové zariadenia vieme pridať kliknutím myši, namiesto montovania do stojana.
\item energetické úspory - Jeden fyzický server je schopný virtualizovať desiatky zariadení.
\item jednoduchšia administrácia - O všetkých zariadeniach máme prehľad z jedného miesta, či už je to webové rozhranie, klientská aplikácia alebo výpis procesov v termináli.
\end{itemize}

Vo svojej práci sa zaoberám štyrmi riešeniami pre virtuálne sieťové laboratórium: EVE-ng, GNS3, Cisco VIRL a Dynamips. Každý z nich dôkladne otestujem podľa vopred stanovených kritérií. Potom v nich vypracujem úlohy pre vybrané predmety vyučované na Katedre informačných sietí: Počítačové siete 1, Počítačové siete 2, Projektovanie sietí 1 a Projektovanie sietí 2. Takto overím vhodnosť týchto nástrojov pre vyučovanie daných predmetov. Nakoniec vyhodnotím výsledky testovania týchto riešení a ich prínos pre výučbu.