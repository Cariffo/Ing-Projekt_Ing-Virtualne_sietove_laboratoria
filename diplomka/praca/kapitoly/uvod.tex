\chapter*{Úvod}
\addcontentsline{toc}{chapter}{Úvod}

Virtualizácia sa stáva vo svete čoraz populárnejšou. Využíva sa v rôznych oblastiach napríklad v tzv. Cloud Computing alebo Software Defined Networking. Virtualizáciou sa zaberá aj Katedra informačných sietí Žilinskej univerzity v Žiline. Jedným z projektov, kde sa virtualizácia využíva, je projekt virtuálneho sieťového laboratória. 

Mojou úlohou je preskúmať existujúce riešenia v oblasti virtuálnych sieťových laboratórii a porovnať ich podľa vopred stanovených kritérií. Na základe toho z nich vyberiem jedno riešenie, ktorým sa budem v práci ďalej zaoberať. Pre vybrané sieťové laboratórium následne vypracujem návody na inštaláciu, úpravu, používanie a nasadenie do infraštruktúry katedry. Následne analyzujem kompatibilitu nástroja s rôznymi zariadeniami. Vybrané zariadenia otestujem, do akej miery vyťažujú systémove zdroje, aby som im mohol nastaviť primerané parametre. Potom si vyberiem predmety, na ktoré má byť tento nástroj použítý a analyzujem technológie vyučované na týchto predmetoch. Nakoniec zistím, ktoré zariadenia sú pre daný predmet vhodné.

Virtuálne sieťové laboratórium bude nástrojom, ktorý zefektívni a skvalitní výučbu sieťových technológii vo vybraných predmetoch na Katedre informačných sietí.

Tému diplomovej práce som si vybral predovšetkým preto, aby som pomohol dosiahnuť tento cieľ: skvalitniť a zefektívniť vyučovací proces na katedre. Tak budú mať učitelia aj študenti jednotnú platformu pre vyučovanie, ktorá učiteľom umožní jednoducho vytvárať modelové situácie a študentom uľahčí ich pochopenie.