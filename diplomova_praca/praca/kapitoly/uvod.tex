\chapter*{Úvod}
\addcontentsline{toc}{chapter}{Úvod}

Virtualizácia sa stáva vo svete čoraz populárnejšou. Využíva sa v rôznych oblastiach napríklad v tzv. Cloud Computing alebo Software Defined Networking. Virtualizáciou sa zaberá aj Katedra informačných sietí Žilinskej univerzity v Žiline. Jedným z projektov, kde sa virtualizácia využíva, je projekt virtuálneho sieťového laboratória. 

Našou úlohou je preskúmať existujúce riešenia v oblasti virtuálnych sieťových laboratórii a porovnať ich podľa vopred stanovených kritérií (kapitola \ref{chap:nastroje_pre_siet_virt}). Na základe toho z nich vyberieme jedno riešenie, ktorým sa budeme v práci ďalej zaoberať. Pre vybrané sieťové laboratórium následne vypracujeme návody na inštaláciu, úpravu, používanie a nasadenie do infraštruktúry katedry (kapitola \ref{chap:eve_ng}). Následne si vyberieme predmety, na ktoré má byť tento nástroj použítý a analyzujeme technológie, ktoré sa na nich vyučujú. Zároveň odhadneme, ktoré zariadenia sú pre daný predmet vhodné (kapitola \ref{chap:analyza_vyucovania}). Potom analyzujeme kompatibilitu nástroja s rôznymi zariadeniami a vybrané zariadenia otestujeme podľa rôznych kritérii (kapitola \ref{chap:virt_zariadenia}). Nakoniec vybrané riešenie pre virtuálne sieťové laboratórium nasadíme do vyučovacieho procesu Katedry informačných sietí (kapitola \ref{chap:nasadenie_do_vyucovania}).

Virtuálne sieťové laboratórium bude nástrojom, ktorý zefektívni a skvalitní výučbu sieťových technológii vo vybraných predmetoch na Katedre informačných sietí.

Téma diplomovej práce bola vybraná predovšetkým preto, aby bola dosiahnutá kvalitnejší a efektívnejší vyučovací proces na katedre. Tak budú mať učitelia aj študenti jednotnú platformu pre vyučovanie, ktorá umožní jednoducho vytvárať modelové situácie pri riešení problémov v sieťovej infraštruktúre.